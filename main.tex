\documentclass{beamer}
\usepackage{listings}
\lstset{
%language=C,
frame=single, 
breaklines=true,
columns=fullflexible
}
\usepackage{subcaption}
\usepackage{url}
\usepackage{tikz}
\usepackage{graphicx}
\usepackage{tkz-euclide} % loads  TikZ and tkz-base
%\usetkzobj{all}
\usetikzlibrary{calc,math}
\usepackage{float}
\newcommand\norm[1]{\left\lVert#1\right\rVert}
\renewcommand{\vec}[1]{\mathbf{#1}}
\newcommand{\R}{\mathbb{R}}
\newcommand{\C}{\mathbb{C}}
\providecommand{\brak}[1]{\ensuremath{\left(#1\right)}}
\providecommand{\abs}[1]{\vert#1\vert}
\providecommand{\fourier}{\overset{\mathcal{F}}{ \rightleftharpoons}}
\providecommand{\pr}[1]{\ensuremath{\Pr\left(#1\right)}}
\providecommand{\sbrak}[1]{\ensuremath{{}\left[#1\right]}}
\usepackage[export]{adjustbox}
\usepackage[utf8]{inputenc}
\usepackage{amsmath}
\usetheme{Boadilla}
\title{GATE 2011(ME) Problem-19}
\author{Sachin Karumanchi (IITH AI)}
\date{AI20BTECH11013}
\begin{document}
%
\begin{frame}
\titlepage
\end{frame}
\begin{frame}{Question}
    \begin{block}{GATE 2011(ME) Problem-19}
    Cars arrive at a service station according to Poisson’s distribution with a mean rate of 5 per hour.The service time per car is exponential with a mean of 10 minutes.At a steady state,the average waiting time in the queue is
    \begin{enumerate}
        \item 10 min
        \item 20 min
        \item 25 min
        \item 50 min
    \end{enumerate}
    \end{block}
\end{frame}
\begin{frame}{Queuing Theory}
\begin{itemize}
    \item Queuing theory is the mathematical study of waiting lines or queues.
    \item In Queuing theory a model is constructed so that queue lengths and waiting times can be predicted.
    \item A customer arrives into system for service and gets out of system after service.
    \item Generally most of the arrival times follow poisson distributions and services follow exponential distributions.
    \item These queuing models have since seen applications including telecommunications,traffic engineering, computing and the design of factories,shops,offices and hospitals. 
\end{itemize}
\end{frame}
\begin{frame}{Parameters for measuring Queuing performance}
\begin{block}{Parameters}
\begin{itemize}
    \item $\lambda$ = Average arrival time
    \item $\mu$ = Average service time
    \item $\rho$ = Utilization factor
    \item $L_q$ = Average number in the queue
    \item $L_s$ = Average number in the system
    \item $W_q$ = Average waiting time
    \item $W_s$ = Average time in the system
    \item $P_n$ = Steady state probability of exactly n customers in the system
\end{itemize}
\end{block}
\end{frame}
\begin{frame}{Single server model}
    \begin{itemize}
        \item From the given question we can say that there is only one queue with no limit of cars in queue,so we can say that it is a single server model.
        \item Single server model is represented as "\textbf{M/M/1/$\infty$/$\infty$/FIFO}" by kendall's notation (or) usually "\textbf{M/M/1}"
        \item  Here '\textbf{M}' indicates the Markovian property (or) memory less property of the model first \textbf{M} is for arrival and second one for service and 1 is the number of servers in the model and '$\infty$' indicates the limit of the queue and second '$\infty$' represent population of jobs to be served and '\textbf{FIFO}' represents First-In First-Out service.
        \item In cases where there is no limit in the queue we only take the cases where $\frac{\lambda}{\mu}<1$. Otherwise there could be customers who will not get their service.
    \end{itemize}
\end{frame}
\begin{frame}{Deriving formulae}
    \begin{itemize}
    \item The memory less property allows us to assume that one event can take place in a small interval of time. The event could be either a arrival or a service.
    \item  For the time interval($t,t+h$), where $h \to 0$
\begin{align}
    \Pr{(\text{1 arrival})}&=\lambda h\\
    \Pr{(\text{1 service})}&=\mu h\\
    \Pr{(\text{no arrival})}&=1-\lambda h\\
    \Pr{(\text{no service})}&=1-\mu h
\end{align}
\item Probability of $n$ people at time $(t+h)$
\begin{multline}
    P_{n}(t+h)=P_{n-1}(t)\times\Pr{(\text{1 arrival})}\times\Pr{(\text{no service})}\\
    +P_{n+1}(t)\times\Pr{(\text{no arrival})}\times\Pr{(\text{1 service})}\\
    +P_n(t)\times\Pr{(\text{no arrival})}\times\Pr{(\text{no service})}\label{eq:eq1}
\end{multline}
    \end{itemize}
\end{frame}
\begin{frame}{contd}
\begin{multline}
    \implies P_{n}(t+h)=P_{n-1}(t)(\lambda h)(1-\mu h)+P_{n+1}(t)(\mu h)(1- \lambda h)\\
    +P_n(t)(1-\lambda h)(1-\mu h)
\end{multline}
Now, Neglecting higher order terms of $h$.
\begin{align}
    \implies P_{n}(t+h)=P_{n-1}\lambda h+P_{n+1}\mu h
    +P_n(t)(1-\lambda h-\mu h)\\
    \implies \frac{P_n(t+h)-P_n(t)}{h}=P_{n-1}(t)\lambda+P_{n+1}(t)\mu
    -P_n(t)(\lambda+\mu)
\end{align}
At steady state, $P_n(t+h)=P_n(t)$
\begin{align}
    \implies\lambda P_{n-1}+\mu P_{n+1}&=(\lambda+\mu)P_n\label{eq:res1}
\end{align}
\end{frame}
\begin{frame}{contd}
Now, calculating $P_0(t+h)$ using \eqref{eq:eq1}
\begin{align}
    P_0(t+h)=P_1(t)(1-\lambda h)(\mu h)
    +P_0(t)(1-\lambda h)
\end{align}
Again, Neglecting higher order terms of $h$
\begin{align}
    \implies P_0(t+h)=P_1(t)(\mu h)+P_0(t)(1-\lambda h)
    \end{align}
    \begin{align}
    \implies\frac{P_0(t+h)-P_0(t)}{h}&=P_1(\mu)-P_0(\lambda)
\end{align}
At steady state, $P_0(t+h)=P_0(t)$
\begin{align}
    \implies \mu P_1&=\lambda P_0\label{eq:eq2}\\
    \implies P_1&=\brak{\frac{\lambda}{\mu}}P_0\label{eq:res2}
\end{align}
\end{frame}
\begin{frame}{contd}
Using \eqref{eq:res1} by substituting $n=1$
\begin{align}
    \lambda P_0+\mu P_2&=(\lambda+\mu)P_1\\
    \implies \lambda P_0+\mu P_2&=\lambda P_1+\mu P_1
\end{align}
from \eqref{eq:eq2} and \eqref{eq:res2}
\begin{align}
    \implies\lambda P_0+\mu P_2&=\lambda P_1+\lambda P_0\\
    \implies P_2&=\brak{\frac{\lambda}{\mu}}P_1\\
    \implies P_2&=\brak{\frac{\lambda}{\mu}}^2P_0\label{eq:eq3}
\end{align}
We assume $\frac{\lambda}{\mu}=\rho$ and generalize $P_n$ by \eqref{eq:res2} and \eqref{eq:eq3}
\begin{align}
  P_n&=\brak{\frac{\lambda}{\mu}}^nP_0\\
  \implies P_n&=\rho^nP_0\label{eq:res3}
\end{align}
\end{frame}
\begin{frame}{contd}
We know that sum of all probabilities equal to 1
\begin{align}
    \sum_{i=1}^{\infty}P_i&=1\\
    \implies P_0+P_1+P_2+....&=1
\end{align}
Using \eqref{eq:res3}
\begin{align}
    \implies P_0+\rho P_0+\rho^2P_0+....&=1\\
    \implies P_0\brak{1+\rho+\rho^2+...}&=1\\
    \implies P_0\brak{\frac{1}{1-\rho}}&=1\\
    \implies P_0&=1-\rho\label{eq:res4}\\
    \therefore P_n=\rho^n(1-\rho)
\end{align}
\end{frame}
\begin{frame}{Contd}
The number of people in the system ($L_s$) is the expected value
\begin{align}
    L_s&=\sum_{i=0}^{\infty}iP_i\\
    \implies L_s&=\sum_{i=0}^{\infty}i\rho^iP_0=\rho P_0\sum_{i=0}^{\infty}i\rho^{i-1}\\
    \implies L_s&=\rho P_0\sum_{i=0}^{\infty}\frac{d}{d\rho}\brak{\rho^i}\\
    \implies L_s&=\rho P_0\frac{d}{d\rho}\sum_{i=0}^{\infty}\rho^i\\
    \implies L_s&=\rho P_0\frac{d}{d\rho}\brak{\frac{1}{1-\rho}}\\
    \implies L_s&=\rho P_0\frac{1}{(1-\rho)^2}
\end{align}
\end{frame}
\begin{frame}{contd}
By using \eqref{eq:res4}
\begin{align}
    \implies L_s&=\rho(1-\rho)\frac{1}{(1-\rho)^2}\\
    \implies L_s&=\frac{\rho}{1-\rho}=\frac{\lambda}{\mu-\lambda}
\end{align}
The average number of people beign served is $\rho$
\begin{align}
    \therefore L_s&=L_q+\text{avg no.of people beign served}\\
    \implies L_s&=L_q+\rho\\
    \implies L_q&=L_s-\rho\\
    \implies L_q&=\frac{\rho}{1-\rho}-\rho\\
    \implies L_q&=\frac{\rho^2}{1-\rho}=\frac{\lambda^2}{\mu(\mu-\lambda)}
\end{align}
\end{frame}
\begin{frame}{contd}
    The relation between $L_s$,$W_s$ and $L_q$,$W_q$ are the Little's equation and they are related as
\begin{align}
    W_s&=\frac{L_s}{\lambda}=\frac{1}{\mu-\lambda}\\
    W_q&=\frac{L_q}{\lambda}=\frac{\lambda}{\mu(\mu-\lambda)}
\end{align}
\begin{block}{Results}
\begin{itemize}
    \item $\rho=\frac{\lambda}{\mu}$
    \item $L_s=\frac{\lambda}{\mu-\lambda}$
    \item $L_q=\frac{\lambda^2}{\mu(\mu-\lambda)}$
    \item $W_s=\frac{1}{\mu-\lambda}$
    \item $W_q=\frac{\lambda}{\mu(\mu-\lambda)}$
\end{itemize}
\end{block}
\end{frame}
\begin{frame}{Solution}
From the question given,
\begin{align*}
    \lambda &=5\text{hr}^{-1}\\
    \mu&=\frac{1}{10}\text{min}^{-1} =6\text{hr}^{-1}
    \end{align*}
also $\rho=\frac{\lambda}{\mu}=\frac{5}{6}<1$\\
Therefore, the average waiting time in the queue 
\begin{align*}
    W_q&=\frac{\lambda}{\mu(\mu-\lambda)}\\
    &=\frac{5}{6(6-5)}\\
    &=\frac{5}{6}\text{hr}=50\text{min}
\end{align*}
So,\textbf{option 4} is correct.
\end{frame}
\end{document}